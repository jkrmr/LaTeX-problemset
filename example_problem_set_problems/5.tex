\problem[Exercise 5.8.8 (15pts)]

Consider the boundary value problem
%
\begin{align}
	\diff[2]{\phi}{x} &+ \lambda\phi = 0 \label{pde}
	\shortintertext{with}
	\phi(0)-\diff{\phi}{x}(0)=0  &\mtxt{and}
  \phi(1)+\diff{\phi}{x}(1)=0.\label{bc}
\end{align}
%
\begin{enumerate}[(a)]
  \item Using the Rayleigh Quotient, show that $\lambda \geq 0$.
        Why is $\lambda > 0$?
  \item Prove that eigenfunctions corresponding to different
        eigenvalues are orthogonal.
  \item Show that \[ \tan \sqrt \lambda = \frac{2\sqrt\lambda}{\lambda-1}. \]
        Determine the eigenvalues graphically. Estimate the large eigenvalues.
  \item Solve
        \begin{align}
          \diffp{u}{t} &= k \diffp[2]{u}{x}
          \shortintertext{with}
          u(0,t)-\diffp{u}{x}(0,t)
          = 0 \mc u(1,t)&+\diffp{u}{x}(1,t)
          = 0  \mtxt[,]{and} u(x,0)=f(x).
        \end{align}
        You may call the relevant eigenfunctions $\phi_n(x)$
        and assume they are known.
\end{enumerate}

\solution{
\begin{enumerate}[(a)]
  \item %(a)
  With $p=1$, $q=0$, $\s=1$, the Rayleigh quotient for \eqref{pde} becomes
  \[\l = \frac{ \eval{-\h\h'}{0}{1} + \Int{[\h']^2}{0}{1} }{ \Int{\h^2}{0}{1} }. \]

  Since the boundary conditions imply that $-\h_1'=\h_1$ and $\h_0'=\h_0$,
   \[\eval{-\h\h'}{0}{1} = [-\h_1 \h'_1] - [-\h_0\h_0'] = [\h_1 \h_1] + [\h_0\h_0] = \h_1^2 + \h_0^2 \geq 0. \]

  Thus \[  \l = \frac{ \h_1^2 + \h_0^2 + \Int{[\h']^2}{0}{1} }{ \Int{\h^2}{0}{1} } \geq 0. \]

  Since $\l=0$ only if $\h_1^2 + \h_0^2 = -\Int{[\h']^2}{0}{1}$,
  which in turn implies the eigenfunction $\h \equiv 0$, a contradiction.
  So $\l>0$.

  \item %(b)
  Orthogonality results directly from noticing that \eqref{pde} subject to the boundary conditions in \eqref{bc}
  is a regular Sturm-Liouville eigenvalue problem. Without relying on other theorems, however, we prove orthogonality by deriving an
  expression that relates any two of its eigenvalues.

  Let $\l_m$ and $\l_n$ be eigenvalues of \eqref{pde} with corresponding eigenfunctions $\h_m$ and $\h_n$. Then we have
  \begin{align}
    \phi_m'' + \lambda_m \phi_m = 0      \label{m}
    \shortintertext{and}
    \phi_n'' + \lambda_n \phi_n = 0.  \label{n}
  \end{align}
  Multiplying \eqref{m} by $\h_n$ and \eqref{n} by $\h_m$ and subtracting yields
  $\h_m''\h_n + \l_m\h_m\h_n - \h_n''\h_m - \l_n\h_n\h_m=0$, so
  \begin{align*}
    \l_m\h_m\h_n - \l_n\h_n\h_m &= \h_n''\h_m -\h_n\h_m''\\
    (\l_m-\l_n)\h_n\h_m &= \h_n''\h_m-\h_n\h_m''.
  \end{align*}
  Integrating, we have
  \begin{align*}
      (\l_m-\l_n) \Int{\h_n\h_m}{0}{1}
      &= \Int{\h_n''\h_m-\h_n\h_m''}{0}{1} \\
      &= \Int{\h_n''\h_m+(\h_n'\h_m'-\h_n'\h_m')-\h_n\h_m''}{0}{1} \\
      &= \Int{(\h_n''\h_m+\h_n'\h_m')-(\h_n'\h_m'+\h_n\h_m'')}{0}{1} \\
      &= \Int{\D{}{x}[\h_n'\h_m]-\D{}{x}[\h_n\h_m']}{0}{1} \\
      &= \Int{\D{}{x}[\h_n'\h_m]}{0}{1} - \Int{\D{}{x}[\h_n\h_m']}{0}{1} \\
      &= \eval{\h_n'\h_m}{0}{1} - \eval{\h_n\h_m'}{0}{1} \\
      &= \eval{\h_n'\h_m-\h_n\h_m'}{0}{1} \\
      &= \Eval{\h_n'\h_m-\h_n\h_m'}{1} - \Eval{\h_n'\h_m-\h_n\h_m'}{0}\\
      &= \Eval{-\h_n\h_m+\h_n\h_m}{1} - \Eval{\h_n\h_m-\h_n\h_m}{0} \tag*{$\L[\text{\small{from the boundary conditions \eqref{bc}}}\R]$}\\
      &= 0
  \end{align*}

  Thus if $m \neq n$ then $\Int{\h_n\h_m}{0}{1}=0$,
  and for arbitrary $m$ and $n$ the eigenfunctions $\h_m$ and $\h_n$ corresponding to
  eigenvalues $\l_m$ and $\l_n$ are orthogonal to each other.$\hfill\qed$

  \item %(d)
  Keeping in mind that $\l>0$, we impose the boundary conditions to the spatial problem's general solution and its derivative
  \begin{align*}
    \h(x) &= c_1 \cos \sl x + c_2 \sin \sl x\\
    \h'(x) &= -c_1 \sl \sin \sl x + c_2 \sl \cos \sl x
  \end{align*}
  to yield $c_1 = c_2 \sl$, implying that the eigenfunction is any multiple of $\h(x)=\sl \cos \sl x + \sin \sl x$.

  Applying the second boundary condition, we have
  \begin{align}
    \h(1) &= \h'(1)  \notag\\
    c_2 \L[\sl \cos \sl + \sin \sl \R] &= -c_2 \L[-\l \sin \sl + \sl \cos \sl \R],\notag
    \shortintertext{and thus}
    2\sl \cos \sl &= (\l - 1) \sin \sl \label{satisfyme}
    \shortintertext{and since $\cos \sl \neq 0$,}
    \tan \sl &= \frac{ 2 \sl }{ \l - 1 } \mtxt[,]{as desired.}\notag
  \end{align}
  Division by $\cos \sl$ is possible because if $\cos \sl = 0$, then $\sin \sl = \pm 1 \neq 0$, in which case \eqref{satisfyme} would not be satisfied, since the regularity of \eqref{pde} and \eqref{bc} ensures there are infinite eigenvalues.


  To determine the eigenvalues graphically, we note that, letting $s=\sl>0$ and $f(s) =\frac{2s}{s^2-1}$, we have \\
  $f'(s) =-\frac{2(s^2+1)}{(s^2-1)^2}<0$. Thus there is a horizontal asymptote at $f(s)\!=\!0$, a vertical asymptote at $s\!=\!1$, \\
  $f(s) \gtrless 0$ when $s \gtrless 1$, and $f$ is everywhere decreasing, as illustrated in Figure~\ref{eigenvalues}.

  \begin{figure}[ht!]
    \centering
    % \usepackage{pgfplots}
% pgfplot settings
% \pgfplotsset{
% 			width=7cm,
% 			compat=newest, 
% 			label style={font=\small},
% 			legend style={font=\small},
% }

% command to create verical asymptotes
\providecommand{\vasymptote}[2][]{
    \draw [densely dashed,#1] ({rel axis cs:0,0} -| {axis cs:#2,0}) -- ({rel axis cs:0,1} -| {axis cs:#2,0});
}

% commands for setting color and lineweight of plots
\providecommand{\tangentcolor}{black}
\providecommand{\functioncolor}{black}
\providecommand{\tangentlineweight}{thick}
\providecommand{\functionlineweight}{ultra thick}

% plot image
 \begin{tikzpicture}
	\tikzset{dot/.style={fill=black,circle,scale=3}}  % node dot settings
	\begin{axis}[
		height=8cm,
		width=13cm,
		grid=none,
		axis x line=center,
		axis y line=left,
		ymin=-6,	ymax=6,
		xmin=0, 	xmax=8,		
		xlabel={$\sqrt{\lambda}$},
		ylabel={\rotatebox{90}{$z$}},
		x label style={at={(1.07,0.45)}},
		y label style={at={(0,0.999)}},
		xtick={0,1,1.5708,3.14159,4.7123889,6.283185307,7.853981634},
		xticklabels={$0$, $1$, $\Frac{\pi}{2}$, $\pi$, $\Frac{3\pi}{2}$, $2\pi$, $\Frac{5\pi}{2}$},
		xticklabel style={inner xsep=1pt,anchor=north east},
		ytick={0},
		legend style={draw=none,at={(0.975,0.15)}},
		]
		% tan(s)
		\addplot[\tangentlineweight,domain=0:(pi/2),samples=101,unbounded coords=jump,color=\tangentcolor,mark=none]{tan(deg(x))};
		% f(s)
		\addplot[\functionlineweight,domain=0:1,samples=101,unbounded coords=jump,color=\functioncolor,mark=none]{(2*x)/(x^2-1)};
		\addplot[\functionlineweight,domain=1:10,samples=101,unbounded coords=jump,color=\functioncolor,mark=none]{(2*x)/(x^2-1)};
		% Legend Entries
		\addlegendentry{$\tan\sqrt{\lambda}$}
	    \addlegendentry{$ ^{2\sqrt{\lambda}}\! /\! _{\lambda-1} $} %\frac{2\sqrt{\lambda}}{\lambda-1}
		% Remaining portions of tan(s)
		\addplot[\tangentlineweight,domain=(pi/2):(3*pi/2),samples=101,unbounded coords=jump,color=\tangentcolor,mark=none]{tan(deg(x))};
		\addplot[\tangentlineweight,domain=(3*pi/2):7.8,samples=101,unbounded coords=jump,color=\tangentcolor,mark=none]{tan(deg(x))};
		% Vertical asymptotes
		\vasymptote{1}
		\vasymptote{1.570796327}
	   	\vasymptote{4.71238898}
		\vasymptote{7.853981634}
		% Nodes for lambda_i
		\node at (axis cs:1.30654,3.6957) [dot,pin={[pin distance=6pt]0:\small{$\sqrt{\lambda_1}$}},inner sep=0pt] {};
		\node at (axis cs:3.68900,.585526) [dot,pin={[pin distance=5pt]100:\small{$\sqrt{\lambda_2}$}},inner sep=0pt]{};
		\node at (axis cs:6.58462,0.3109) [dot,pin={[pin distance=5pt]100:\small{$\sqrt{\lambda_3}$}},inner sep=0pt]{};	
    \end{axis}
\end{tikzpicture}
    \caption{Graphical determination of eigenvalues.}
    \label{eigenvalues}
  \end{figure}

  We note from the figure that the graphs of $\tan \sl$ and $\frac{ 2 \sl }{ \l-1 }$ meet between 1 and \Frac{\pi}{2} and that as $n$ grows, solutions approach those of $\tan \sl = 0$. That is, as $n \to \infty$, $\l_n \to [(n-1)\pi]^2$. Thus for large $n$ we have eigenvalues $\l_n \sim (n-1)^2\pi^2$.

  \item %(e)
  From the preceding work, we know the problem in \eqref{pde} and \eqref{bc} has a product solution
   \[  u(x,t) = \infsum{1} a_n \L( \sl[_n] \cos \sl[_n]x + \sin \sl[_n]x \R) e^{-\l_nt} \]
  such that
  \[  \tan \sl[_n] = \frac{2\sl[_n]}{\l_n-1} \]
  with $a_n$ determined by imposing the initial condition and simplifying by appeal to the orthogonality established in (b), yielding
  \[ a_n = \frac{\Int{f(x)\h_n(x)}{0}{1} }{ \Int{\h^2_n(x)}{0}{1}  }.   \]
\end{enumerate}
}
